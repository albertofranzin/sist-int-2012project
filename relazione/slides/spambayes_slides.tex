%%%%%%%%%%%%%%%%%%%%%%%%%%%%%%%%%%%%%%%%%%%%%%%%%%%%%%%%%%%%
%%  This Beamer template was created by Cameron Bracken.
%%  Anyone can freely use or modify it for any purpose
%%  without attribution.
%%
%%  Last Modified: January 9, 2009
%%

\documentclass[xcolor=x11names,compress]{beamer}

%% General document %%%%%%%%%%%%%%%%%%%%%%%%%%%%%%%%%%
\usepackage{graphicx}
\usepackage{tikz}
\usetikzlibrary{decorations.fractals}
%%%%%%%%%%%%%%%%%%%%%%%%%%%%%%%%%%%%%%%%%%%%%%%%%%%%%%


%% Beamer Layout %%%%%%%%%%%%%%%%%%%%%%%%%%%%%%%%%%
\useoutertheme[subsection=false,shadow]{miniframes}
\useinnertheme{default}
\usefonttheme{serif}
\usepackage{palatino}

\setbeamerfont{title like}{shape=\scshape}
\setbeamerfont{frametitle}{shape=\scshape}

\setbeamercolor*{lower separation line head}{bg=DeepSkyBlue4} 
\setbeamercolor*{normal text}{fg=black,bg=white} 
\setbeamercolor*{alerted text}{fg=red} 
\setbeamercolor*{example text}{fg=black} 
\setbeamercolor*{structure}{fg=black} 
 
\setbeamercolor*{palette tertiary}{fg=black,bg=black!10} 
\setbeamercolor*{palette quaternary}{fg=black,bg=black!10} 

\renewcommand{\(}{\begin{columns}}
\renewcommand{\)}{\end{columns}}
\newcommand{\<}[1]{\begin{column}{#1}}
\renewcommand{\>}{\end{column}}
%%%%%%%%%%%%%%%%%%%%%%%%%%%%%%%%%%%%%%%%%%%%%%%%%%


\title{A spam classifier based on Bayes network}
%\author{\shortstack{Alberto Franzin\\1012883}, \shortstack{Fabio Palese\\1033575}}
\author{Alberto Franzin, Fabio Palese}
\date{January 15th, 2013}
\institute[2013]{Sistemi Intelligenti}


\begin{document}


%%%%%%%%%%%%%%%%%%%%%%%%%%%%%%%%%%%%%%%%%%%%%%%%%%%%%%
%%%%%%%%%%%%%%%%%%%%%%%%%%%%%%%%%%%%%%%%%%%%%%%%%%%%%%
% \section{\scshape Introduction}
% \begin{frame}
% \title{A spam classifier based on Bayes network}
% %\subtitle{SUBTITLE}
% \author{
% 	Alberto Franzin, Fabio Palese\\
% 	{\it Humboldt State University}\\
% }
% \date{
% 	\begin{tikzpicture}[decoration=Koch curve type 2] 
% 		\draw[DeepSkyBlue4] decorate{ decorate{ decorate{ (0,0) -- (3,0) }}}; 
% 	\end{tikzpicture}  
% 	\\
% 	\vspace{1cm}
% 	\today
% }
% \titlepage
% \end{frame}

\frame{\titlepage}

% \section[Outline]{}
% \frame{\tableofcontents}

%%%%%%%%%%%%%%%%%%%%%%%%%%%%%%%%%%%%%%%%%%%%%%%%%%%%%%
%%%%%%%%%%%%%%%%%%%%%%%%%%%%%%%%%%%%%%%%%%%%%%%%%%%%%%
\begin{frame}{Introduction}
\tableofcontents
\end{frame}

%%%%%%%%%%%%%%%%%%%%%%%%%%%%%%%%%%%%%%%%%%%%%%%%%%%%%%
%%%%%%%%%%%%%%%%%%%%%%%%%%%%%%%%%%%%%%%%%%%%%%%%%%%%%%
% \section{\scshape Background}
% \subsection{frame 1}
% \begin{frame}{frame 1}
% \begin{itemize}
% \item Item A
% \item Item B
% \begin{itemize}
% \item Subitem 1
% \item Subtem 2
% \end{itemize}
% \item Item C
% \end{itemize}
% \end{frame}

% %%%%%%%%%%%%%%%%%%%%%%%%%%%%%%%%%%%%%%%%%%%%%%%%%%%%%%
% %%%%%%%%%%%%%%%%%%%%%%%%%%%%%%%%%%%%%%%%%%%%%%%%%%%%%%
% \subsection{frame 2}
% \begin{frame}{frame 2}

% \end{frame}

% %%%%%%%%%%%%%%%%%%%%%%%%%%%%%%%%%%%%%%%%%%%%%%%%%%%%%%
% %%%%%%%%%%%%%%%%%%%%%%%%%%%%%%%%%%%%%%%%%%%%%%%%%%%%%%
% \subsection{frame 3}
% \begin{frame}{frame 3}

% \end{frame}

%%%%%%%%%%%%%%%%%%%%%%%%%%%%%%%%%%%%%%%%%%%%%%%%%%%%%%
%%%%%%%%%%%%%%%%%%%%%%%%%%%%%%%%%%%%%%%%%%%%%%%%%%%%%%

\section{Introduction}

\frame {
    \frametitle{The Bayesian approach}
    Bayes rule:
          $$
          P(A|B) = \frac{P(A \cap B)}{P(B)} = \frac{P(B|A)P(A)}{P(B)}
          $$
    \begin{itemize}
        \item<1->$P(A|B)$ is the \textit{a posteriori} probability of event $A$, knowing the event $B$ has already occurred.
        \item<2->$P(B|A)$ is the \textit{likelihood}
        \item<3->$P(B|A)P(A)$ is the \textit{prior} probability
        \item<4->$P(B) = \sum_{a \in A}P(B|A=a)P(A=a)$ is the \textit{total} probability
    \end{itemize}
}

\frame {
    \frametitle{The Bayesian approach}
    Bayes rule:
        $$
        P(A|B) = \frac{P(A \cap B)}{P(B)} = \frac{P(B|A)P(A)}{P(B)}
        $$
    \begin{itemize}
        \item<1->In other words, we can estimate the probability of an hypothesis, given that we know the consequences.
        \item<2->This has led to two different interpretations of the theorem.
    \end{itemize}
}

\frame {
    \frametitle{The Bayesian approach}
    Frequentists vs. Bayesians
    \only<1>{
        \begin{center}
            \pgfimage[height=5cm]{fvs1}

            \tiny from http://xkcd.com/1132, see also\\http://en.wikipedia.org/wiki/Sunrise\_problem
        \end{center}
    }
    \only<2>{
        \begin{center}
            \pgfimage[height=4cm]{fvs2}

            \tiny from http://xkcd.com/1132, see also\\http://en.wikipedia.org/wiki/Sunrise\_problem

            \normalsize The frequentist relies on the theoretical probability of the events.
        \end{center}
    }
    \only<3>{
        \begin{center}
            \pgfimage[height=4cm]{fvs3}

            \tiny from http://xkcd.com/1132, see also\\http://en.wikipedia.org/wiki/Sunrise\_problem
 
            \normalsize The bayesian observes the past events occurred,\\
            and adapts the probability accordingly.
        \end{center}
    }
}

\section{Bayesian networks}

\subsection{Definition}

\frame {
    \frametitle{Bayesian networks}
    A Bayesian network is a way to describe causal relationships between events (J. Pearl, 1985).
      \begin{itemize}
          \item<1->Nodes = events
          \item<1->(Directed) Edges = causal relationship
          \item<2->Two nodes are connected by an edge: the child of an arc is influenced by its ancestor in a probabilistic way
      \end{itemize}
%     \only<3->{
%         \begin{itemize}
%             \item<3->This will only appear on the second page
%             \item<3->This is also only for the second page
%         \end{itemize}
%       }
}

\frame {
    \frametitle{An example}
      \begin{center}
          \pgfimage[height=6cm]{bayes_network}
      \end{center}
}

%%%%%%%%%%%%%%%%%%%%%%%%%%%%%%%%%%%%%%%%%%%%%%%%%%%%%%
%%%%%%%%%%%%%%%%%%%%%%%%%%%%%%%%%%%%%%%%%%%%%%%%%%%%%%
% %\subsection{Conditional Independence}
% \begin{frame}{Conditional independence}
%   \begin{itemize}
%     \item<1->{If
%     $$
%     P(A|B,C) = P(A|B)
%     $$
%     then we say that $B$ and $C$ are \textit{conditionally independent}.}
%     \item<2->{
%         Note that \textit{conditional independence} $\neq$ \textit{independence}
%     }
%     \item<3->Explaining away: if we know that one possible cause of the event has happened, this may \textit{explain away} the event, being all the other causes less probable once we know the one that happened.
%   \end{itemize}
% \end{frame}

%%%%%%%%%%%%%%%%%%%%%%%%%%%%%%%%%%%%%%%%%%%%%%%%%%%%%%
%%%%%%%%%%%%%%%%%%%%%%%%%%%%%%%%%%%%%%%%%%%%%%%%%%%%%%
\subsection{Naive Bayes}
\begin{frame}{Naive Bayes}
\begin{itemize}
    \item<1->Computing all the probabilities in a Bayesian network requires exponential time.
      We introduce the assumption of independence among variables.
    \item<2->It is called \textit{naive}, since it's often unrealistic, but it yields good results.
    \item<3->In spam classification:
      \begin{center}
        \pgfimage[height=3cm]{bayes_network_spam}
      \end{center}
\end{itemize}
\end{frame}

%%%%%%%%%%%%%%%%%%%%%%%%%%%%%%%%%%%%%%%%%%%%%%%%%%%%%%
%%%%%%%%%%%%%%%%%%%%%%%%%%%%%%%%%%%%%%%%%%%%%%%%%%%%%%
\subsection{Naive Bayes for spam classification}
\begin{frame}{Naive Bayes for spam classification}
We want to build a classifier that distinguishes good mails from undesired mails:
\begin{itemize}
    \item good mails: \textit{ham}
    \item undesired mails: \textit{spam}
\end{itemize}
\end{frame}

%%%%%%%%%%%%%%%%%%%%%%%%%%%%%%%%%%%%%%%%%%%%%%%%%%%%%%
%%%%%%%%%%%%%%%%%%%%%%%%%%%%%%%%%%%%%%%%%%%%%%%%%%%%%%
\begin{frame}{Algorithm}
Algorithm:
\begin{itemize}
    \item<1->Training
    \item<2->Validation
    \item<3->Testing
\end{itemize}
\end{frame}

%%%%%%%%%%%%%%%%%%%%%%%%%%%%%%%%%%%%%%%%%%%%%%%%%%%%%%
%%%%%%%%%%%%%%%%%%%%%%%%%%%%%%%%%%%%%%%%%%%%%%%%%%%%%%
\begin{frame}{Naive Bayes for spam classification}
Formulas:
\begin{itemize}
    \item<1->For each word: $$P_{word|spam} = \frac{\mbox{\# occurrences of word in spam mails}}{\mbox{\# total occurrences of word}}$$
    \item<2->Final for spam is: $$P_{spam} = \prod_{words \in mail} P_{spam|word}$$
    \item<3->same for ham
    \item<4->Outcome is the class that maximizes the probability of belonging to that class.
\end{itemize}
\end{frame}

%%%%%%%%%%%%%%%%%%%%%%%%%%%%%%%%%%%%%%%%%%%%%%%%%%%%%%
%%%%%%%%%%%%%%%%%%%%%%%%%%%%%%%%%%%%%%%%%%%%%%%%%%%%%%
\section{SpamBayes}
\begin{frame}{SpamBayes}
  \begin{itemize}
    %\item<1->
    \item SpamBayes: applying Naive Bayes to spam classification
    \item Python with Ply and BeautifulSoup
    \item dataset: SpamAssassin archive
    \item Code and documentation available at \url{http://code.google.com/p/sist-int-2012project/}\\
  \end{itemize}
\end{frame}

%%%%%%%%%%%%%%%%%%%%%%%%%%%%%%%%%%%%%%%%%%%%%%%%%%%%%%
%%%%%%%%%%%%%%%%%%%%%%%%%%%%%%%%%%%%%%%%%%%%%%%%%%%%%%

\begin{frame}{SpamBayes}
  \begin{center}
    \pgfimage[height=6cm]{uml}
  \end{center}
\end{frame}

%%%%%%%%%%%%%%%%%%%%%%%%%%%%%%%%%%%%%%%%%%%%%%%%%%%%%%
%%%%%%%%%%%%%%%%%%%%%%%%%%%%%%%%%%%%%%%%%%%%%%%%%%%%%%
\subsection{Implementation}
\begin{frame}{Notes on implementation}
\begin{itemize}
    \item<1->Smoothing:
      $$P_{word|spam} = \frac{\mbox{\# occurrences of word in spam mails} + k}{\mbox{\# total occurrences of the word} + |C|\times k}$$
    \item<2->Calculations can be simplified: some words bring little contribution to the mail status
    \item<3->Several mail features detected
    \item<4->Several parameters to be tuned: we describe the more relevant ones
\end{itemize}
\end{frame}

%%%%%%%%%%%%%%%%%%%%%%%%%%%%%%%%%%%%%%%%%%%%%%%%%%%%%%
%%%%%%%%%%%%%%%%%%%%%%%%%%%%%%%%%%%%%%%%%%%%%%%%%%%%%%
\section{Tests}
\begin{frame}{Parameters}
%\begin{itemize}
    \only<1>{Size of training/validation/test sets:
        \begin{center}
          \pgfimage[height=3cm]{size_bags}
          \pgfimage[height=3cm]{size_test}
        \end{center}
    }
    \only<2>{Relevance threshold:
        \begin{center}
          \pgfimage[height=4cm]{rel_threshold}
        \end{center}
    }
    \only<3>{``Spamicity'' threshold:
        \begin{center}
          \pgfimage[height=4cm]{spam_thr}
        \end{center}}
    \only<4>{Feature statistic threshold:
        \begin{center}
          \pgfimage[height=4cm]{feats_threshold_1}
        \end{center}}
%\end{itemize}
\end{frame}

%%%%%%%%%%%%%%%%%%%%%%%%%%%%%%%%%%%%%%%%%%%%%%%%%%%%%%
%%%%%%%%%%%%%%%%%%%%%%%%%%%%%%%%%%%%%%%%%%%%%%%%%%%%%%
\section{Results and Conclusions}
%\subsection{About the dataset}
\begin{frame}{Results}
  \begin{itemize}
    \item<1->About the dataset:
      \begin{itemize}
        \item<2->General features
        \item<3->Single words
      \end{itemize}
    \item<4->About the classification:
      \begin{itemize}
        \item<5->Accuracy
        \item<6->How to improve?
      \end{itemize}
  \end{itemize}
\end{frame}

%%%%%%%%%%%%%%%%%%%%%%%%%%%%%%%%%%%%%%%%%%%%%%%%%%%%%%
%%%%%%%%%%%%%%%%%%%%%%%%%%%%%%%%%%%%%%%%%%%%%%%%%%%%%%
% \begin{frame}{Final}
%   \begin{center}
%     Any questions?
%   \end{center}
% \end{frame}

\end{document}
