% Generated by Sphinx.
\def\sphinxdocclass{report}
\documentclass[letterpaper,10pt,english]{sphinxmanual}
\usepackage[utf8]{inputenc}
\DeclareUnicodeCharacter{00A0}{\nobreakspace}
\usepackage[T1]{fontenc}
\usepackage{babel}
\usepackage{times}
\usepackage[Bjarne]{fncychap}
\usepackage{longtable}
\usepackage{sphinx}
\usepackage{multirow}


\title{SpamBayes Documentation}
\date{December 12, 2012}
\release{0.1}
\author{Alberto Franzin, Fabio Palese}
\newcommand{\sphinxlogo}{}
\renewcommand{\releasename}{Release}
\makeindex

\makeatletter
\def\PYG@reset{\let\PYG@it=\relax \let\PYG@bf=\relax%
    \let\PYG@ul=\relax \let\PYG@tc=\relax%
    \let\PYG@bc=\relax \let\PYG@ff=\relax}
\def\PYG@tok#1{\csname PYG@tok@#1\endcsname}
\def\PYG@toks#1+{\ifx\relax#1\empty\else%
    \PYG@tok{#1}\expandafter\PYG@toks\fi}
\def\PYG@do#1{\PYG@bc{\PYG@tc{\PYG@ul{%
    \PYG@it{\PYG@bf{\PYG@ff{#1}}}}}}}
\def\PYG#1#2{\PYG@reset\PYG@toks#1+\relax+\PYG@do{#2}}

\def\PYG@tok@gd{\def\PYG@tc##1{\textcolor[rgb]{0.63,0.00,0.00}{##1}}}
\def\PYG@tok@gu{\let\PYG@bf=\textbf\def\PYG@tc##1{\textcolor[rgb]{0.50,0.00,0.50}{##1}}}
\def\PYG@tok@gt{\def\PYG@tc##1{\textcolor[rgb]{0.00,0.25,0.82}{##1}}}
\def\PYG@tok@gs{\let\PYG@bf=\textbf}
\def\PYG@tok@gr{\def\PYG@tc##1{\textcolor[rgb]{1.00,0.00,0.00}{##1}}}
\def\PYG@tok@cm{\let\PYG@it=\textit\def\PYG@tc##1{\textcolor[rgb]{0.25,0.50,0.50}{##1}}}
\def\PYG@tok@vg{\def\PYG@tc##1{\textcolor[rgb]{0.10,0.09,0.49}{##1}}}
\def\PYG@tok@m{\def\PYG@tc##1{\textcolor[rgb]{0.40,0.40,0.40}{##1}}}
\def\PYG@tok@mh{\def\PYG@tc##1{\textcolor[rgb]{0.40,0.40,0.40}{##1}}}
\def\PYG@tok@go{\def\PYG@tc##1{\textcolor[rgb]{0.50,0.50,0.50}{##1}}}
\def\PYG@tok@ge{\let\PYG@it=\textit}
\def\PYG@tok@vc{\def\PYG@tc##1{\textcolor[rgb]{0.10,0.09,0.49}{##1}}}
\def\PYG@tok@il{\def\PYG@tc##1{\textcolor[rgb]{0.40,0.40,0.40}{##1}}}
\def\PYG@tok@cs{\let\PYG@it=\textit\def\PYG@tc##1{\textcolor[rgb]{0.25,0.50,0.50}{##1}}}
\def\PYG@tok@cp{\def\PYG@tc##1{\textcolor[rgb]{0.74,0.48,0.00}{##1}}}
\def\PYG@tok@gi{\def\PYG@tc##1{\textcolor[rgb]{0.00,0.63,0.00}{##1}}}
\def\PYG@tok@gh{\let\PYG@bf=\textbf\def\PYG@tc##1{\textcolor[rgb]{0.00,0.00,0.50}{##1}}}
\def\PYG@tok@ni{\let\PYG@bf=\textbf\def\PYG@tc##1{\textcolor[rgb]{0.60,0.60,0.60}{##1}}}
\def\PYG@tok@nl{\def\PYG@tc##1{\textcolor[rgb]{0.63,0.63,0.00}{##1}}}
\def\PYG@tok@nn{\let\PYG@bf=\textbf\def\PYG@tc##1{\textcolor[rgb]{0.00,0.00,1.00}{##1}}}
\def\PYG@tok@no{\def\PYG@tc##1{\textcolor[rgb]{0.53,0.00,0.00}{##1}}}
\def\PYG@tok@na{\def\PYG@tc##1{\textcolor[rgb]{0.49,0.56,0.16}{##1}}}
\def\PYG@tok@nb{\def\PYG@tc##1{\textcolor[rgb]{0.00,0.50,0.00}{##1}}}
\def\PYG@tok@nc{\let\PYG@bf=\textbf\def\PYG@tc##1{\textcolor[rgb]{0.00,0.00,1.00}{##1}}}
\def\PYG@tok@nd{\def\PYG@tc##1{\textcolor[rgb]{0.67,0.13,1.00}{##1}}}
\def\PYG@tok@ne{\let\PYG@bf=\textbf\def\PYG@tc##1{\textcolor[rgb]{0.82,0.25,0.23}{##1}}}
\def\PYG@tok@nf{\def\PYG@tc##1{\textcolor[rgb]{0.00,0.00,1.00}{##1}}}
\def\PYG@tok@si{\let\PYG@bf=\textbf\def\PYG@tc##1{\textcolor[rgb]{0.73,0.40,0.53}{##1}}}
\def\PYG@tok@s2{\def\PYG@tc##1{\textcolor[rgb]{0.73,0.13,0.13}{##1}}}
\def\PYG@tok@vi{\def\PYG@tc##1{\textcolor[rgb]{0.10,0.09,0.49}{##1}}}
\def\PYG@tok@nt{\let\PYG@bf=\textbf\def\PYG@tc##1{\textcolor[rgb]{0.00,0.50,0.00}{##1}}}
\def\PYG@tok@nv{\def\PYG@tc##1{\textcolor[rgb]{0.10,0.09,0.49}{##1}}}
\def\PYG@tok@s1{\def\PYG@tc##1{\textcolor[rgb]{0.73,0.13,0.13}{##1}}}
\def\PYG@tok@sh{\def\PYG@tc##1{\textcolor[rgb]{0.73,0.13,0.13}{##1}}}
\def\PYG@tok@sc{\def\PYG@tc##1{\textcolor[rgb]{0.73,0.13,0.13}{##1}}}
\def\PYG@tok@sx{\def\PYG@tc##1{\textcolor[rgb]{0.00,0.50,0.00}{##1}}}
\def\PYG@tok@bp{\def\PYG@tc##1{\textcolor[rgb]{0.00,0.50,0.00}{##1}}}
\def\PYG@tok@c1{\let\PYG@it=\textit\def\PYG@tc##1{\textcolor[rgb]{0.25,0.50,0.50}{##1}}}
\def\PYG@tok@kc{\let\PYG@bf=\textbf\def\PYG@tc##1{\textcolor[rgb]{0.00,0.50,0.00}{##1}}}
\def\PYG@tok@c{\let\PYG@it=\textit\def\PYG@tc##1{\textcolor[rgb]{0.25,0.50,0.50}{##1}}}
\def\PYG@tok@mf{\def\PYG@tc##1{\textcolor[rgb]{0.40,0.40,0.40}{##1}}}
\def\PYG@tok@err{\def\PYG@bc##1{\fcolorbox[rgb]{1.00,0.00,0.00}{1,1,1}{##1}}}
\def\PYG@tok@kd{\let\PYG@bf=\textbf\def\PYG@tc##1{\textcolor[rgb]{0.00,0.50,0.00}{##1}}}
\def\PYG@tok@ss{\def\PYG@tc##1{\textcolor[rgb]{0.10,0.09,0.49}{##1}}}
\def\PYG@tok@sr{\def\PYG@tc##1{\textcolor[rgb]{0.73,0.40,0.53}{##1}}}
\def\PYG@tok@mo{\def\PYG@tc##1{\textcolor[rgb]{0.40,0.40,0.40}{##1}}}
\def\PYG@tok@kn{\let\PYG@bf=\textbf\def\PYG@tc##1{\textcolor[rgb]{0.00,0.50,0.00}{##1}}}
\def\PYG@tok@mi{\def\PYG@tc##1{\textcolor[rgb]{0.40,0.40,0.40}{##1}}}
\def\PYG@tok@gp{\let\PYG@bf=\textbf\def\PYG@tc##1{\textcolor[rgb]{0.00,0.00,0.50}{##1}}}
\def\PYG@tok@o{\def\PYG@tc##1{\textcolor[rgb]{0.40,0.40,0.40}{##1}}}
\def\PYG@tok@kr{\let\PYG@bf=\textbf\def\PYG@tc##1{\textcolor[rgb]{0.00,0.50,0.00}{##1}}}
\def\PYG@tok@s{\def\PYG@tc##1{\textcolor[rgb]{0.73,0.13,0.13}{##1}}}
\def\PYG@tok@kp{\def\PYG@tc##1{\textcolor[rgb]{0.00,0.50,0.00}{##1}}}
\def\PYG@tok@w{\def\PYG@tc##1{\textcolor[rgb]{0.73,0.73,0.73}{##1}}}
\def\PYG@tok@kt{\def\PYG@tc##1{\textcolor[rgb]{0.69,0.00,0.25}{##1}}}
\def\PYG@tok@ow{\let\PYG@bf=\textbf\def\PYG@tc##1{\textcolor[rgb]{0.67,0.13,1.00}{##1}}}
\def\PYG@tok@sb{\def\PYG@tc##1{\textcolor[rgb]{0.73,0.13,0.13}{##1}}}
\def\PYG@tok@k{\let\PYG@bf=\textbf\def\PYG@tc##1{\textcolor[rgb]{0.00,0.50,0.00}{##1}}}
\def\PYG@tok@se{\let\PYG@bf=\textbf\def\PYG@tc##1{\textcolor[rgb]{0.73,0.40,0.13}{##1}}}
\def\PYG@tok@sd{\let\PYG@it=\textit\def\PYG@tc##1{\textcolor[rgb]{0.73,0.13,0.13}{##1}}}

\def\PYGZbs{\char`\\}
\def\PYGZus{\char`\_}
\def\PYGZob{\char`\{}
\def\PYGZcb{\char`\}}
\def\PYGZca{\char`\^}
\def\PYGZsh{\char`\#}
\def\PYGZpc{\char`\%}
\def\PYGZdl{\char`\$}
\def\PYGZti{\char`\~}
% for compatibility with earlier versions
\def\PYGZat{@}
\def\PYGZlb{[}
\def\PYGZrb{]}
\makeatother

\begin{document}

\maketitle
\tableofcontents
\phantomsection\label{index::doc}



\chapter{Main module}
\label{index:main-module}\label{index:spambayes-s-documentation}
To launch the program, open a terminal, go to hell and type

\emph{python spam\_bayes.py}

Bye.
\phantomsection\label{index:module-spam_bayes}\index{spam\_bayes (module)}

\chapter{The Bayes network definition}
\label{index:the-bayes-network-definition}

\section{The Naive Bayes class}
\label{index:module-naive_bayes}\label{index:the-naive-bayes-class}\index{naive\_bayes (module)}\index{Bayes (class in naive\_bayes)}

\begin{fulllineitems}
\phantomsection\label{index:naive_bayes.Bayes}\pysigline{\strong{class }\code{naive\_bayes.}\bfcode{Bayes}}
Contains the Bayes network and some possible operations: training,
validation, k-fold cross-validation, formatted print of the data.
For the other operations, instantiate the apposite classes.
\index{\_\_init\_\_() (naive\_bayes.Bayes method)}

\begin{fulllineitems}
\phantomsection\label{index:naive_bayes.Bayes.__init__}\pysiglinewithargsret{\bfcode{\_\_init\_\_}}{}{}
Constructor.

Initialize all the objects and variables used to define a Bayes network:
words stats, overall stats, configuration, trainer, validator.
Saves the path of the project.

\end{fulllineitems}

\index{\_k\_fold\_cross\_validation() (naive\_bayes.Bayes method)}

\begin{fulllineitems}
\phantomsection\label{index:naive_bayes.Bayes._k_fold_cross_validation}\pysiglinewithargsret{\bfcode{\_k\_fold\_cross\_validation}}{\emph{spam\_list}, \emph{ham\_list}}{}
Internal method, execute the k-fold cross-validation TODO: FINISH

Splits the lists in the desidered number of parts
(see {\hyperref[index:config.Config]{\code{config.Config}}} object),
then calls the {\hyperref[index:trainer.Trainer.train]{\code{trainer.Trainer.train()}}} function.
\begin{quote}\begin{description}
\item[{Parameters}] \leavevmode\begin{itemize}
\item {} 
\textbf{spam\_list} (\emph{array of str}) -- the list of spam mails to be used;

\item {} 
\textbf{ham\_list} (\emph{array of str}) -- the list of ham mails to be used;

\end{itemize}

\item[{Returns}] \leavevmode
the accuracy of the training.

\end{description}\end{quote}

\end{fulllineitems}

\index{bayes\_print() (naive\_bayes.Bayes method)}

\begin{fulllineitems}
\phantomsection\label{index:naive_bayes.Bayes.bayes_print}\pysiglinewithargsret{\bfcode{bayes\_print}}{\emph{print\_words}, \emph{print\_gen\_stats}}{}
Prints out the data, padded for alignment.

Slightly adapted from \href{http://ginstrom.com/scribbles/2007/09/04/pretty-printing-a-table-in-python/}{http://ginstrom.com/scribbles/2007/09/04/pretty-printing-a-table-in-python/}, many thanks.

Each row must have the same number of columns.
\begin{quote}\begin{description}
\item[{Parameters}] \leavevmode\begin{itemize}
\item {} 
\textbf{print\_words} (\emph{bool}) -- do I have to print the words retrieved?

\item {} 
\textbf{print\_gen\_stats} (\emph{bool}) -- do I have to print the overall stats?

\end{itemize}

\end{description}\end{quote}

\end{fulllineitems}

\index{test\_bayes() (naive\_bayes.Bayes method)}

\begin{fulllineitems}
\phantomsection\label{index:naive_bayes.Bayes.test_bayes}\pysiglinewithargsret{\bfcode{test\_bayes}}{}{}
Performs some test - needed to try some functions.

\end{fulllineitems}

\index{train() (naive\_bayes.Bayes method)}

\begin{fulllineitems}
\phantomsection\label{index:naive_bayes.Bayes.train}\pysiglinewithargsret{\bfcode{train}}{}{}
Train the net. TODO: COMPLETE CROSS-VALIDATION

Read the mails given as training and validation set for spam and ham,
then executes the proper training. Two methods are available: the direct
training, and the k-fold cross-validation.

First of all, read the training set and validation set mails.
If the k-fold cross-validation is chosen (see {\hyperref[index:config.Config]{\code{config.Config}}}
documentation), then call the apposite method, otherwise calls the
{\hyperref[index:trainer.Trainer]{\code{trainer.Trainer}}} object to extract from the training set
the feature stats, then compute the accuracy by calling the
{\hyperref[index:naive_bayes.Bayes.validate]{\code{naive\_bayes.Bayes.validate()}}} object, to find out the goodness of
the classification.

No parameters are needed, since everything the network needs is already
present. The location of the mails is (for now?) hardcoded here.

\end{fulllineitems}

\index{validate() (naive\_bayes.Bayes method)}

\begin{fulllineitems}
\phantomsection\label{index:naive_bayes.Bayes.validate}\pysiglinewithargsret{\bfcode{validate}}{\emph{ham\_val\_list}, \emph{spam\_val\_list}, \emph{words}, \emph{general\_stats}, \emph{config}}{}
Validation function.

Get the validation sets and the results of the training, and
compute the accuracy of the classification of the mails
in the validation set.
\begin{quote}\begin{description}
\item[{Parameters}] \leavevmode\begin{itemize}
\item {} 
\textbf{ham\_val\_list} (\emph{array of mails}) -- the good mails of the validation set;

\item {} 
\textbf{spam\_val\_list} (\emph{array of mails}) -- the spam mails of the validation set;

\item {} 
\textbf{words} (array of {\hyperref[index:gen_stat.Word]{\code{gen\_stat.Word}}} objects) -- the list of words read so far, and their stats;

\item {} 
\textbf{general\_stats} (associative array \{str, {\hyperref[index:gen_stat.Stat]{\code{gen\_stat.Stat}}}\}) -- the overall stats of the features;

\item {} 
\textbf{config} ({\hyperref[index:config.Config]{\code{config.Config}}} object) -- contains some general parameters and configurations;

\end{itemize}

\item[{Returns}] \leavevmode
accuracy of the validation.

\end{description}\end{quote}

\end{fulllineitems}


\end{fulllineitems}



\section{The configuration options manager}
\label{index:module-config}\label{index:the-configuration-options-manager}\index{config (module)}\index{Config (class in config)}

\begin{fulllineitems}
\phantomsection\label{index:config.Config}\pysigline{\strong{class }\code{config.}\bfcode{Config}}
Contains some general configurations.

The available parameters are (with \emph{{[}default{]}} values):
\begin{itemize}
\item {} 
CROSS\_VALIDATION (bool): True if k-fold cross-validation is chosen.        False otherwise {[}True{]};

\item {} 
CROSS\_VALIDATION\_FOLDS (int): the number of folds for        cross-validation, if enabled {[}4{]};

\item {} 
OVERALL\_FEATS\_SPAM\_W (float, in {[}0,1{]}): the weight of the overall stats        when computing the spamicity of a mail. The remaining part is given by        the word stats {[}0.7{]};

\item {} 
SHORT\_THR (int): length of a word to be identified as \emph{very short} {[}1{]};

\item {} 
SIZE\_OF\_BAGS (int): number of ham and spam mails for training {[}50{]};

\item {} 
SIZE\_OF\_VAL\_BAGS (int): number of ham and spam mails for validation {[}10{]};

\item {} 
SMOOTH\_VALUE (int): smoothing value to be used in classification {[}1{]};

\item {} 
SPAM\_THR (float, in {[}0,1{]}): probability threshold to mark a mail as spam {[}0.95{]};

\item {} 
VERBOSE (bool): if True, displays more messages {[}True{]};

\item {} 
VERYLONG\_THR (int): length of a word to be identified as \emph{very long} {[}20{]}.

\end{itemize}
\index{\_\_init\_\_() (config.Config method)}

\begin{fulllineitems}
\phantomsection\label{index:config.Config.__init__}\pysiglinewithargsret{\bfcode{\_\_init\_\_}}{}{}
Constructor. Initialize all the parameters to their default value.

\end{fulllineitems}


\end{fulllineitems}



\section{The training class}
\label{index:the-training-class}\label{index:module-trainer}\index{trainer (module)}\index{Trainer (class in trainer)}

\begin{fulllineitems}
\phantomsection\label{index:trainer.Trainer}\pysigline{\strong{class }\code{trainer.}\bfcode{Trainer}}
Trains the network, computing the stats for the main features
and for the single words.
\index{\_\_init\_\_() (trainer.Trainer method)}

\begin{fulllineitems}
\phantomsection\label{index:trainer.Trainer.__init__}\pysiglinewithargsret{\bfcode{\_\_init\_\_}}{}{}
Constructor.

\end{fulllineitems}

\index{train() (trainer.Trainer method)}

\begin{fulllineitems}
\phantomsection\label{index:trainer.Trainer.train}\pysiglinewithargsret{\bfcode{train}}{\emph{mails}, \emph{is\_spam}, \emph{words}, \emph{general\_stats}, \emph{config}}{}
The proper trainer method.

For all the mails given, extract the single words and classify them,
calculating the overall stats for some interesting features to be
evaluated, and for the single words.
\begin{quote}\begin{description}
\item[{Parameters}] \leavevmode\begin{itemize}
\item {} 
\textbf{mails} (\emph{array of str}) -- the list of mails.

\item {} 
\textbf{is\_spam} (\emph{bool}) -- are the given mails spam?

\item {} 
\textbf{words} (\emph{array of Word objects}) -- the array of stats for the single words detected.

\item {} 
\textbf{general\_stats} (array of \{str, {\hyperref[index:gen_stat.Stat]{\code{gen\_stat.Stat}}}\}) -- the overall stats of the set.

\item {} 
\textbf{config} ({\hyperref[index:config.Config]{\code{config.Config}}} object) -- contains some configurations.

\end{itemize}

\end{description}\end{quote}

\end{fulllineitems}

\index{trainer\_print() (trainer.Trainer method)}

\begin{fulllineitems}
\phantomsection\label{index:trainer.Trainer.trainer_print}\pysiglinewithargsret{\bfcode{trainer\_print}}{\emph{general\_stats}}{}
Print out the overall stats given. For test purposes.
\begin{quote}\begin{description}
\item[{Parameters}] \leavevmode
\textbf{general\_stats} (array of \{str, {\hyperref[index:gen_stat.Stat]{\code{gen\_stat.Stat}}}\}) -- the overall stats to be printed.

\end{description}\end{quote}

\end{fulllineitems}


\end{fulllineitems}



\section{The classifier class}
\label{index:the-classifier-class}\label{index:module-classifier}\index{classifier (module)}\index{Classifier (class in classifier)}

\begin{fulllineitems}
\phantomsection\label{index:classifier.Classifier}\pysigline{\strong{class }\code{classifier.}\bfcode{Classifier}}
Classify the test set.

Apply Bayesian logic to classify the mails as spam or ham.
\index{classify() (classifier.Classifier static method)}

\begin{fulllineitems}
\phantomsection\label{index:classifier.Classifier.classify}\pysiglinewithargsret{\strong{static }\bfcode{classify}}{\emph{mws}, \emph{mgs}, \emph{words}, \emph{general\_stats}, \emph{config}}{}
TODO: INSERT DOCUMENTATION HERE!!!

\end{fulllineitems}


\end{fulllineitems}



\chapter{Feature statistics modules}
\label{index:feature-statistics-modules}

\section{General stats for training sets}
\label{index:general-stats-for-training-sets}\label{index:module-gen_stat}\index{gen\_stat (module)}\index{Stat (class in gen\_stat)}

\begin{fulllineitems}
\phantomsection\label{index:gen_stat.Stat}\pysiglinewithargsret{\strong{class }\code{gen\_stat.}\bfcode{Stat}}{\emph{description}, \emph{words\_spam}, \emph{words\_ham}}{}
Stats for mail characteristics: how many times this feature appears in
a spam mail, and how many times it appears in a ham mail. Class used when
training the network.
\index{\_\_init\_\_() (gen\_stat.Stat method)}

\begin{fulllineitems}
\phantomsection\label{index:gen_stat.Stat.__init__}\pysiglinewithargsret{\bfcode{\_\_init\_\_}}{\emph{description}, \emph{words\_spam}, \emph{words\_ham}}{}
Constructor.

\end{fulllineitems}


\end{fulllineitems}

\index{Word (class in gen\_stat)}

\begin{fulllineitems}
\phantomsection\label{index:gen_stat.Word}\pysiglinewithargsret{\strong{class }\code{gen\_stat.}\bfcode{Word}}{\emph{spam\_occurrences}, \emph{ham\_occurrences}}{}
Stats for a single word: how many times this word appears in a spam
mail, and how many times it appears in a ham mail. Class used when
training the network.
\index{\_\_init\_\_() (gen\_stat.Word method)}

\begin{fulllineitems}
\phantomsection\label{index:gen_stat.Word.__init__}\pysiglinewithargsret{\bfcode{\_\_init\_\_}}{\emph{spam\_occurrences}, \emph{ham\_occurrences}}{}
Constructor.

\end{fulllineitems}


\end{fulllineitems}



\section{General stats for validation and test sets}
\label{index:module-test_stat}\label{index:general-stats-for-validation-and-test-sets}\index{test\_stat (module)}\index{Test\_stat (class in test\_stat)}

\begin{fulllineitems}
\phantomsection\label{index:test_stat.Test_stat}\pysiglinewithargsret{\strong{class }\code{test\_stat.}\bfcode{Test\_stat}}{\emph{description}, \emph{count}}{}
Stats for a single mail belonging to the test set or to
the validation set. So, it it not possible, at the stage this object
is created, to tell whether the mail is spam or ham. Class used when
validating and testing the network.
\index{\_\_init\_\_() (test\_stat.Test\_stat method)}

\begin{fulllineitems}
\phantomsection\label{index:test_stat.Test_stat.__init__}\pysiglinewithargsret{\bfcode{\_\_init\_\_}}{\emph{description}, \emph{count}}{}
Constructor. Initialize the stat.

\end{fulllineitems}


\end{fulllineitems}

\index{Test\_word (class in test\_stat)}

\begin{fulllineitems}
\phantomsection\label{index:test_stat.Test_word}\pysiglinewithargsret{\strong{class }\code{test\_stat.}\bfcode{Test\_word}}{\emph{occurrences}}{}
Stats for a single word: how many times this word appears in the parsed
mail. Class used when validating and testing the network.
\index{\_\_init\_\_() (test\_stat.Test\_word method)}

\begin{fulllineitems}
\phantomsection\label{index:test_stat.Test_word.__init__}\pysiglinewithargsret{\bfcode{\_\_init\_\_}}{\emph{occurrences}}{}
Constructor.

\end{fulllineitems}


\end{fulllineitems}



\chapter{Various tools and utilities}
\label{index:various-tools-and-utilities}

\section{The lexical analyzer}
\label{index:the-lexical-analyzer}\label{index:module-lexer}\index{lexer (module)}\index{Lexer (class in lexer)}

\begin{fulllineitems}
\phantomsection\label{index:lexer.Lexer}\pysigline{\strong{class }\code{lexer.}\bfcode{Lexer}}
Lexical Analyzer. Use Ply's lexer to identify the tokens and to classify them.
See \href{http://www.dabeaz.com/ply/}{http://www.dabeaz.com/ply/} to know how it works.
\index{\_\_init\_\_() (lexer.Lexer method)}

\begin{fulllineitems}
\phantomsection\label{index:lexer.Lexer.__init__}\pysiglinewithargsret{\bfcode{\_\_init\_\_}}{}{}
Constructor: creates the \emph{Ply} lexer and defines all the rules to identify
and classify the tokens.

All the \code{t\_TOKEN()} methods are defined as inner methods inside here.

\end{fulllineitems}

\index{\_process\_tokens() (lexer.Lexer method)}

\begin{fulllineitems}
\phantomsection\label{index:lexer.Lexer._process_tokens}\pysiglinewithargsret{\bfcode{\_process\_tokens}}{\emph{results}, \emph{in\_training}, \emph{is\_spam}, \emph{words}, \emph{general\_stats}, \emph{config}}{}
Process tokens extracted from the training set.

For every token, extract the value (the word itself)
and its type (lowercase word, title, link, etc),
then update all the stats for the word and the mail.
\begin{quote}\begin{description}
\item[{Parameters}] \leavevmode\begin{itemize}
\item {} 
\textbf{results} (\emph{array of tokens}) -- the list of tokens recognized;

\item {} 
\textbf{in\_training} -- flag to tell if the lexing is performed during training            (\emph{True}) or during validation or testing (\emph{False}). If we are performing            the training step, then we know if the mail processed is ham or spam, and
so we can fill appropriately the \emph{general\_stats} array of
{\hyperref[index:gen_stat.Stat]{\code{gen\_stat.Stat}}}, otherwise the array will be filled with
\code{mail\_stat.Mail\_stat} objects;

\item {} 
\textbf{is\_spam} (\emph{bool}) -- flag to identify the mail as spam or ham (useless if             \emph{in\_training == False});

\item {} 
\textbf{words} (array of {\hyperref[index:gen_stat.Word]{\code{gen\_stat.Word}}} objects) -- the list of words read so far, and their stats;

\item {} 
\textbf{general\_stats} -- the overall stats of the features. Feature type may be            of two types:                {\hyperref[index:gen_stat.Stat]{\code{gen\_stat.Stat}}} (\emph{in\_training == True}), or                \code{mail\_stat.Mail\_stat} (\emph{in\_training == False});

\item {} 
\textbf{config} ({\hyperref[index:config.Config]{\code{config.Config}}} object) -- contains some general parameters and configurations.

\end{itemize}

\end{description}\end{quote}

\end{fulllineitems}

\index{lexer\_words() (lexer.Lexer method)}

\begin{fulllineitems}
\phantomsection\label{index:lexer.Lexer.lexer_words}\pysiglinewithargsret{\bfcode{lexer\_words}}{\emph{text}, \emph{in\_training}, \emph{is\_spam}, \emph{words}, \emph{general\_stats}, \emph{config}}{}
Apply lexical analysis to the text of mails.

May
\begin{quote}\begin{description}
\item[{Parameters}] \leavevmode\begin{itemize}
\item {} 
\textbf{text} (\emph{str}) -- the text of the mail to be parsed;

\item {} 
\textbf{in\_training} -- flag to tell if the lexing is performed during training            (\emph{True}) or during validation or testing (\emph{False}). If we are performing            the training step, then we know if the mail processed is ham or spam, and
so we can fill appropriately the \emph{general\_stats} array of
{\hyperref[index:gen_stat.Stat]{\code{gen\_stat.Stat}}}, otherwise the array will be filled with
\code{mail\_stat.Mail\_stat} objects;

\item {} 
\textbf{is\_spam} (\emph{bool}) -- flag to identify the mail as spam or ham (useless if             \emph{in\_training == False});

\item {} 
\textbf{words} (array of {\hyperref[index:gen_stat.Word]{\code{gen\_stat.Word}}} objects) -- the list of words read so far, and their stats;

\item {} 
\textbf{general\_stats} -- the overall stats of the features. Feature type may be            of two types:                {\hyperref[index:gen_stat.Stat]{\code{gen\_stat.Stat}}} (\emph{in\_training == True}), or                \code{mail\_stat.Mail\_stat} (\emph{in\_training == False});

\item {} 
\textbf{config} ({\hyperref[index:config.Config]{\code{config.Config}}} object) -- contains some general parameters and configurations.

\end{itemize}

\end{description}\end{quote}

\end{fulllineitems}


\end{fulllineitems}



\section{Other utilities}
\label{index:other-utilities}\label{index:module-utils}\index{utils (module)}\index{Utils (class in utils)}

\begin{fulllineitems}
\phantomsection\label{index:utils.Utils}\pysigline{\strong{class }\code{utils.}\bfcode{Utils}}
Collection of various tools used in the project.
\index{\_read\_files() (utils.Utils static method)}

\begin{fulllineitems}
\phantomsection\label{index:utils.Utils._read_files}\pysiglinewithargsret{\strong{static }\bfcode{\_read\_files}}{\emph{path}, \emph{how\_many}, \emph{read\_mails}, \emph{words}, \emph{gen\_stats}, \emph{config}}{}
Read the desidered number of text files from the given path.

If desidered, extract first the text and then the tokens
from the mails. Does nothing on the content of plain text files.
\begin{quote}\begin{description}
\item[{Parameters}] \leavevmode
\textbf{path} -- the relative path from the current position

\end{description}\end{quote}

to the desidered directory;
:type path: str
:param how\_many: how many files to read. 0 == unlimited;
:type how\_many: int
:param read\_mails: tells if the user wants to read mails or plain text;
:type read\_mails: bool
:param words: the list of words read so far, and their stats;
:type words: array of {\hyperref[index:gen_stat.Word]{\code{gen\_stat.Word}}} objects
:param general\_stats: the overall stats of the features;
:type general\_stats: associative array \{str, {\hyperref[index:gen_stat.Stat]{\code{gen\_stat.Stat}}}\}
:param config: contains some general parameters and configurations;
:type config: {\hyperref[index:config.Config]{\code{config.Config}}} object
:return: a list containing all the mails in the given files.

\end{fulllineitems}

\index{chunks() (utils.Utils static method)}

\begin{fulllineitems}
\phantomsection\label{index:utils.Utils.chunks}\pysiglinewithargsret{\strong{static }\bfcode{chunks}}{\emph{l}, \emph{n}}{}
Yield successive n-sized chunks from l.

From \href{http://stackoverflow.com/questions/312443}{http://stackoverflow.com/questions/312443} (thanks).
\begin{quote}\begin{description}
\item[{Parameters}] \leavevmode\begin{itemize}
\item {} 
\textbf{l} (\emph{list of objects}) -- the list to be splitted;

\item {} 
\textbf{n} (\emph{int}) -- the size of the generated chunks.

\end{itemize}

\end{description}\end{quote}

\end{fulllineitems}

\index{create\_stats() (utils.Utils static method)}

\begin{fulllineitems}
\phantomsection\label{index:utils.Utils.create_stats}\pysiglinewithargsret{\strong{static }\bfcode{create\_stats}}{}{}
Defines a new associative array of (str, Stat), containing all the
overall stats to be evaluated by the Bayes network.
\begin{quote}\begin{description}
\item[{Returns}] \leavevmode
the newly created array.

\end{description}\end{quote}

\end{fulllineitems}

\index{read\_mails() (utils.Utils static method)}

\begin{fulllineitems}
\phantomsection\label{index:utils.Utils.read_mails}\pysiglinewithargsret{\strong{static }\bfcode{read\_mails}}{\emph{path}, \emph{how\_many}, \emph{words}, \emph{general\_stats}, \emph{config}}{}
Read the desidered number of text files from the given path.

Calls method Utils.\_read\_files, passing the same parameters received,
with \emph{read\_mails} flag set to True.
\begin{quote}\begin{description}
\item[{Parameters}] \leavevmode\begin{itemize}
\item {} 
\textbf{path} (\emph{str}) -- the relative path from the current position             to the desidered directory;

\item {} 
\textbf{how\_many} (\emph{int}) -- how many files to read. 0 == unlimited;

\item {} 
\textbf{words} (array of {\hyperref[index:gen_stat.Word]{\code{gen\_stat.Word}}} objects) -- the list of words read so far, and their stats;

\item {} 
\textbf{general\_stats} (associative array \{str, {\hyperref[index:gen_stat.Stat]{\code{gen\_stat.Stat}}}\}) -- the overall stats of the features;

\item {} 
\textbf{config} ({\hyperref[index:config.Config]{\code{config.Config}}} object) -- contains some general parameters and configurations;

\end{itemize}

\item[{Returns}] \leavevmode
a list containing all the mails in the given files.

\end{description}\end{quote}

\end{fulllineitems}

\index{read\_text() (utils.Utils static method)}

\begin{fulllineitems}
\phantomsection\label{index:utils.Utils.read_text}\pysiglinewithargsret{\strong{static }\bfcode{read\_text}}{\emph{path}, \emph{how\_many}, \emph{config}}{}
Read the desidered number of text files from the given path.

Calls method Utils.\_read\_files, passing the same parameters received,
with \emph{read\_mails} flag set to False.
\begin{quote}\begin{description}
\item[{Parameters}] \leavevmode\begin{itemize}
\item {} 
\textbf{path} (\emph{str}) -- the relative path from the current position             to the desidered directory;

\item {} 
\textbf{how\_many} (\emph{int}) -- how many files to read. 0 = unlimited;

\item {} 
\textbf{config} ({\hyperref[index:config.Config]{\code{config.Config}}} object;) -- contains some general parameters and configurations;

\end{itemize}

\item[{Returns}] \leavevmode
a list containing all the text in the given files.

\end{description}\end{quote}

\end{fulllineitems}


\end{fulllineitems}



\renewcommand{\indexname}{Python Module Index}
\begin{theindex}
\def\bigletter#1{{\Large\sffamily#1}\nopagebreak\vspace{1mm}}
\bigletter{c}
\item {\texttt{classifier}}, \pageref{index:module-classifier}
\item {\texttt{config}}, \pageref{index:module-config}
\indexspace
\bigletter{g}
\item {\texttt{gen\_stat}}, \pageref{index:module-gen_stat}
\indexspace
\bigletter{l}
\item {\texttt{lexer}}, \pageref{index:module-lexer}
\indexspace
\bigletter{n}
\item {\texttt{naive\_bayes}}, \pageref{index:module-naive_bayes}
\indexspace
\bigletter{s}
\item {\texttt{spam\_bayes}}, \pageref{index:module-spam_bayes}
\indexspace
\bigletter{t}
\item {\texttt{test\_stat}}, \pageref{index:module-test_stat}
\item {\texttt{trainer}}, \pageref{index:module-trainer}
\indexspace
\bigletter{u}
\item {\texttt{utils}}, \pageref{index:module-utils}
\end{theindex}

\renewcommand{\indexname}{Index}
\printindex
\end{document}
